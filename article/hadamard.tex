\documentclass{article}

\author{TAMDRARI Yacine}
\title{Le code de Hadamard}

\begin{document}

\maketitle

\section{Definition}
Le code de Hadamard nommé d'aprés le mathématicien Français Jacques 
Hadamard est un code correcteur d'erreur linéaire utilisé dans plusieurs 
domaines tels que le traitement de signal, le cryptage et la 
télécommunication; il a été utilisé le 14 novembre 1971 pendant la mission
du visseau spataial Mariner 9 pour corriger les erreurs de transmission des
images depuis l'orbite de Mars.

Le code de Hadamard est un code sur un alphabet de taille $2^k$ ou k est un
entier positif, chaque mot du code a une longueur de k bits donc pour coder
une information avec k bits le code ajoute $2^k$-k bits pour former un mot
de code sur $2^k$ bits, ceci implique que pour transférer k bits 
d'informations avec ce code le taux de transfert sera /frac{k}{$2^k$} qui 
est extrêmement faible donc rends le temps de transmission des informations
lent, mais le code est couramment utilisé à cause de sa capacité à détecter
toutes les erreurs simples car la distance de Hamming (le nombre de bits 
différents entre deux mots de code binaires qui peut etre trouvé en sommant
les résultats des XOR bit à bit des deux mots de code) entre deux séquences
de code est $\frac{n}{2}$ = $\frac{2^k}{2}$ = $2^{k-1}$ cela signifie que 
le code serait capable de détecter jusqu'à $2^{k-1}$ erreurs simples, un 
code correcteur à une distance de Hamming d est capable de corriger 
$\frac{d-1}{2}$ erreurs ce qui donne $\frac{2^{k-1}-1}{2}$ pour le code de 
Hadamard (pour avoir une correction il faut que k > 2 donc ce code serait 
fiable sur des canaux trés bruyants sur lequels le risque de corruption 
d'une information est trés grand; dans la notation standard de la théorie 
du codage pour les codes en bloc on note ceci avec $[2^k, k, 2^{k-1}]_2$

\end{document}
