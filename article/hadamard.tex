\documentclass{article}

\author{TAMDRARI Yacine}
\title{Le code de Hadamard}

\begin{document}

\maketitle
\titlepage
\tableofcontents
\newpage

\section{Definition}
Le code de Hadamard nommé d'aprés le mathématicien Français Jacques 
Hadamard est un code correcteur d'erreur linéaire (le plus connu mais il 
existe des codes de Hadamard non linéaire) utilisé dans plusieurs 
domaines tels que le traitement de signal, le cryptage et la 
télécommunication; il a été utilisé le 14 novembre 1971 pendant la mission
du visseau spataial Mariner 9 pour corriger les erreurs de transmission des
images depuis l'orbite de Mars.

Le code de Hadamard est un code sur un alphabet de taille $2^k$ ou k est un
entier positif, chaque mot du code a une longueur de k bits donc pour coder
une information avec k bits le code ajoute $2^k$-k bits pour former un mot
de code sur $2^k$ bits, ceci implique que pour transférer k bits 
d'informations avec ce code le taux de transfert sera /frac{k}{$2^k$} qui 
est extrêmement faible donc rends le temps de transmission des informations
lent, mais le code est couramment utilisé à cause de sa capacité à détecter
toutes les erreurs simples car la distance de Hamming\footnote{le nombre 
de bits différents entre deux mots de code binaires qui peut etre trouvé en
sommant les résultats des XOR bit à bit des deux mots de code} entre deux séquences de code est $\frac{n}{2}$ = $\frac{2^k}{2}$ = $2^{k-1}$ cela 
signifie que le code serait capable de détecter jusqu'à $2^{k-1}$ erreurs 
simples, un code correcteur à une distance de Hamming d est capable de 
corriger $\frac{d-1}{2}$ erreurs ce qui donne $\frac{2^{k-1}-1}{2}$ pour le
code de Hadamard (pour avoir une correction il faut que k > 2 donc ce code 
serait fiable sur des canaux trés bruyants sur lequels le risque de 
corruption d'une information est trés grand; dans la notation standard de 
la théorie du codage pour les codes en bloc on note ceci avec 
$[2^k, k, 2^{k-1}]_2$.

\section{Un peu d'histoire}
Le code de Hadamard est le nom le plus couramment utilisé pour ce code dans
la littérature bien que Jacques Hadamard n'a pas inventé le code de 
lui-meme mais il a initié les bases de ce code en déffinissant les matrices
d'Hadamard vers 1893, bien avant que le premier code correcteur d'erreurs
(le code de Hamming vers 1940) ne soit développé.

Il existe plusieurs construction des matrices d'Hadamard mais la plus 
utilisée est celle définie par James Joseph Sylvester en 1867 qui est en 
fait antérieure aux travaux d'Hadamard sur ses matrices, des codes 
d'Hadamard non linéaires ont été construit par R. C. Bose et S. S. 
Shrikhande en 1959.

Un code Hadamard a été utilisé lors de la mission Mariner 9 de 1971 pour
corriger les erreurs de transmission d'images; les mots de données 
utilisés au cours de cette mission étaient de 6 bits, ce qui représentait 
64 valeurs en niveaux de gris.En raison des limitations de la qualité de 
l'alignement de l'émetteur, la longueur maximale des données utiles était 
environ 30 bits.Au lieu d'utiliser un code de répition, un code Hadamard
a été utilisé.Des erreurs allant jusqu'à 7 bits par mot pourraient etre 
corrigées à l'aide de ce code; par rapport à un code à 5 répétitions, les
propriétés de correction d'erreurs de ce code Hadamard sont bien meilleures
mais son taux est comparable.L’efficacité de l’algorithme de décodage a été
un facteur important dans la décision d’utiliser ce code.Le circuit 
utilisé s'appelait la \textbf{machine verte}.Il utilisait la transformée de
Fourier rapide qui peut augmenter la vitesse de décodage d'un facteur 3.
Depuis les années 1990, l'utilisation de ce code par les programmes 
spatiaux a plus ou moins cessé, et le Deep Space Network ne prend pas en 
charge ce système de correction d'erreurs pour ses antennes paraboliques.
mesurent plus de 26 m.

\end{document}
